\documentclass{article}
\title{Backwards Cookbook}
\author{Calvin Todorovich and Coulby Nguyen}


\date{1/15/2018}

\begin{document}
\maketitle

\section{Introduction}

Cooking is a vital necessity in life, but a large portion of people don't know what to do with the food they have. For example, everybody at some point has looked in the fridge when they were hungry, found nothing, and came back minutes later for the same purpose. This is because they don't know what options they have with the food that's in the fridge, but they don't want to use a cookbook, because they could finally find something that looks good and find out it requires three more ingredients they don't have.  Furthermore, according to marketwatch.com, cooking knowledge has dropped in the next generation, which creates need for a convenient way to learn new recipes. In situations like these, the Backwards Cookbook would solve your problems. This mobile app has the user enter the ingredients present, and it displays all the possible recipes that can be made. Looking up recipes in reverse like this ensures that you can accomplish whatever recipe there is on the screen, by having only the ingredients you provide. 


\section{Body}

At a high level, our application includes accessing a database to get all the recipes we want to provide to the user. Having a thorough database will give the user a wide range of options. 

There is a website that does a service like Backwards Cookbook, but no mobile app. Said website is not a major one, and has a rudimentary interface, which leaves room for competition. We strive to have a much more user friendly interface that will make the app accessible, and easy to use. 

One feature that makes this app effective is the process of looking up recipes in reverse. This eliminates the need to flip through pages of cookbooks, and, instead, scroll over recipes you can make until you find something that looks good. Another feature we could have is another input from the user for the types of food they like best. A feature like that would have recipes more likely to be chosen by the user at the top of the screen. 

One potential limitation includes the range of recipes we can have on the app. With the sheer amount of recipes in the world, and considering we need to have instructions and ingredients as well, a simple search could take up a lot of memory. Another potential limitation 

The resources we will need include android studio or XCode to develop the app, which are free. Another resource we will need is a database that holds the recipes, which we can create or acquire from recipes on-line. 

The database for the recipes is the biggest challenge for this application. Unless we can find a place to pull the data from, we will need to rely on cooking websites for information, which could potentially cause legal trouble. 

To mitigate that risk, we would need to look for a way to reliably use other cooking websites, or find a database that has the information we need. It seems like there are free databases with cooking recipes using MySQL. I believe it is a feasible project to complete with the time constraint. 

\section {Conclusion}

With little competition, and a feasible goal, as long as we get the database working, the Backwards Cookbook is a solid project for completion in 10 weeks. 

\end{document} 